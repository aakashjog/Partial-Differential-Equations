\documentclass[titlepage, fleqn, a4paper, 12pt, twoside]{article}
\usepackage{geometry}
\usepackage{exsheets} %question and solution environments
\usepackage{amsmath, amssymb, amsthm} %standard AMS packages
\usepackage{esint} %integral signs
\usepackage{marginnote} %marginnotes
\usepackage{gensymb} %miscellaneous symbols
\usepackage{commath} %differential symbols
\usepackage{xcolor} %colours
\usepackage{cancel} %cancelling terms
\usepackage[free-standing-units]{siunitx} %formatting units
\usepackage{tikz, pgfplots} %diagrams
	\usetikzlibrary{calc, hobby, patterns, intersections, angles, quotes, spy}
\usepackage{graphicx} %inserting graphics
\usepackage{epstopdf} %converting and inserting eps graphics
\usepackage{hyperref} %hyperlinks
\usepackage{datetime} %date and time
\usepackage{enumerate, enumitem} %numbered lists
\usepackage{float} %inserting floats
\usepackage{microtype} %micro-typography
\usepackage{todonotes}
\usepackage{booktabs}

\newcommand\numberthis{\addtocounter{equation}{1}\tag{\theequation}} %adds numbers to specific equations in non-numbered list of equations

\theoremstyle{definition}
\newtheorem{example}{Example}
\newtheorem{definition}{Definition}

\theoremstyle{theorem}
\newtheorem{theorem}{Theorem}
\newtheorem{law}{Law}

\makeatletter
\@addtoreset{section}{part} %resets section numbers in new part
\makeatother

\newcommand\blfootnote[1]{%
	\begingroup
	\renewcommand\thefootnote{}\footnote{#1}%
	\addtocounter{footnote}{-1}%
	\endgroup
}

\renewcommand{\marginfont}{\scriptsize \color{blue}}

\renewcommand{\tilde}{\widetilde}

\SetupExSheets{solution/print = true} %prints all solutions by default

%opening
\title{Partial Differential Equations}
\author{Aakash Jog}
\date{2015-16}

\begin{document}

\pagenumbering{roman}
\begin{titlepage}
\newgeometry{margin=0cm}
\maketitle
\end{titlepage}
\restoregeometry
%\setlength{\mathindent}{0pt}

\blfootnote
{	
	\begin{figure}[H]
		\includegraphics[height = 12pt]{cc.pdf}
		\includegraphics[height = 12pt]{by.pdf}
		\includegraphics[height = 12pt]{nc.pdf}
		\includegraphics[height = 12pt]{sa.pdf}
	\end{figure}
	This work is licensed under the Creative Commons Attribution-NonCommercial-ShareAlike 4.0 International License. To view a copy of this license, visit \url{http://creativecommons.org/licenses/by-nc-sa/4.0/}.
} %CC-BY-NC-SA license

\tableofcontents

\clearpage
\section{Lecturer Information}

\textbf{Prof. Yakov Yakubov}\\
~\\
Office: Schreiber 233\\
E-mail: \href{mailto:yakubov@post.tau.ac.il}{yakubov@post.tau.ac.il}\\

\section{Recommended Reading}

\begin{enumerate}
	\item Tikhonov, A.N. and Samarskii, N.A: Equations of Mathematical Physics, Pergamon Press, Oxfort, 1963.
	\item Weinberger, H.F, A first Course in Partial Differential Equations, Dover, NY, 1995.
\end{enumerate}

\clearpage
\pagenumbering{arabic}

\part{Introduction}

\section{String Equations}

\begin{definition}[Partial differential equation]
	An equation
	\begin{align*}
		F(x_1,x_2,\dots,u,u_{x_1},u_{x_2},\dots,u_{x_1 x_2},\dots) & = 0
	\end{align*}
	where all $x_i$ are independent variables, and $u(x_1,..,x_n)$ is an unknown function, is called a partial differential equation.\\
	A partial differential equation describes a connection between an unknown function of several variables and its partial derivatives.
\end{definition}

\begin{definition}[Order of a PDE]
	The order of a PDE is defined to be the highest order of partial derivatives in the equation.
\end{definition}

\begin{definition}[Linear PDE]
	A PDE is said o be linear if and only if it is a linear function of $u$ and its partial derivatives.
\end{definition}

\begin{definition}[String equation/1D Wave Equation]
	Consider an ideal string on the $x$-axis.
	Let the string oscillate in the direction normal to the $x$-axis.
	Let $u$ be the position function of a point on the string.
	Therefore, $u$ depends on the position of the point on the string and on the time, i.e. it is a function of $x$ and $t$.\\
	Therefore, solving using Newton's Laws,
	\begin{align*}
		\rho(x) u_{t t}(x,t) & = T u_{x x}(x,t)
	\end{align*}
	where $\rho$ is the mass density of the string, and $T$ is the tension in the string.\\
	If
	\begin{align*}
		\rho(x_0) & = \rho_0
	\end{align*}
	then,
	\begin{align*}
		u_{t t}(x,t) & = a^2 u_{x x}(x,t)
	\end{align*}
	where
	\begin{align*}
		a & = \sqrt{\frac{T}{\rho_0}}
	\end{align*}
	If there is an external force applied to the string,
	\begin{align*}
		\rho(x) u_{t t}(x,t) & = a^2 u_{x x}(x,t) + F(x,t)
	\end{align*}
\end{definition}

\begin{definition}[Cauchy problem]
	Consider an infinite string, i.e. $x \in (-\infty,\infty)$.
	If the initial position and the initial velocity of the string are given to be $f(x)$ and $g(x)$ respectively, then,
	\begin{align*}
		u(x,0)   & = f(x) \\
		u_x(x,0) & = g(x)
	\end{align*}
	The problem
	\begin{align*}
		u_{t t}(x,t) & = a^2 u_{x x}(x,t) \\
		u(x,0)       & = f(x)             \\
		u_x(x,0)     & = g(x)
	\end{align*}
	is called the Cauchy problem.
\end{definition}

\begin{definition}[Dirichlet's boundary conditions]
	Consider a finite string, such that $x \in [0,l]$.
	If the ends of the string are fixed, the boundary conditions
	\begin{align*}
		u(0,t) & = 0 \\
		u(l,t) & = 0
	\end{align*}
	are called Dirichlet's boundary conditions.
\end{definition}

\begin{definition}[General string equation]
	Consider a PDE
	\begin{align*}
		u_{t t}(x,t) & = a^2 u_{x x}(x,t)
	\end{align*}
	Let
	\begin{align*}
		\zeta & = x - a t \\
		\eta  & = x + a t
	\end{align*}
	Therefore,
	\begin{align*}
		u(x,t) & = F(\zeta) + G(\eta) \\
                       & = F(x - a t) + G(x + a t)
	\end{align*}
	where $F$ and $G$ are functions of a single variable, and are differentiable twice.
\end{definition}

\subsection{Infinite Strings}

\begin{theorem}[Solution to Cauchy Problem (Infinite String)]
	The solution to the Cauchy problem
	\begin{align*}
		u_{t t}(x,t) & = a^2 u_{x x}(x,t) \\
		u(x,0)       & = f(x)             \\
		u_x(x,0)     & = g(x)
	\end{align*}
	where $-\infty < x < \infty$, $t \ge 0$ is given by the d'Alembert formula, i.e.
	\begin{align*}
		u(x,t) & = \frac{f(x - a t) + f(x + a t)}{2} + \frac{1}{2 a} \int\limits_{x - a t}^{x + a t} g(s) \dif s
	\end{align*}
	where $f$ is twice differentiable and $g$ is differentiable.
	\label{thm:Solution_to_Cauchy_Problem}
\end{theorem}

\begin{proof}
\begin{align*}
	u_{t t}(x,t) & = a^2 u_{x x}(x,t) \\
	u(x,0)       & = f(x)             \\
	u_x(x,0)     & = g(x)
\end{align*}
Let the solution be
\begin{align*}
	u(x,t) & = F(x - a t) + G(x + a t)
\end{align*}
Therefore,
\begin{align*}
	u_t(x,t) & = \dod{u(x,t)}{(x - a t)} \dod{(x - a t)}{t} \\
                 & = F'(x - a t) (-a) + G'(x + a t) (a)         \\
                 & = -a F'(x - a t) + a G'(x + a t)
\end{align*}
Substituting the initial conditions,
\begin{align*}
	u(x,0)   & = f(x)        \\
                 & = F(x) + G(x) \\
	u_t(x,0) & = g(x)        \\
                 & = -a F'(x) + a G'(x)
\end{align*}
Therefore,
\begin{align*}
	a \int\limits_{0}^{x} \left( -F'(s) + G'(s) \right) \dif s & = \int\limits_{0}^{x} g(s) \dif s \\
	\therefore -F(x) + G(x)                                    & = \frac{1}{a} \int\limits_{0}^{x} g(s) \dif s + c
\end{align*}
Therefore, solving with the initial conditions corresponding to $u(x,0)$,
\begin{align*}
	2 G(x)          & = f(x) + \frac{1}{a} \int\limits_{0}^{x} g(s) \dif s + c                       \\
	\therefore G(x) & = \frac{f(x)}{2} + \frac{1}{2 a} \int\limits_{0}^{x} g(s) \dif s + \frac{c}{2} \\
	2 F(x)          & = f(x) - \frac{1}{a} \int\limits_{0}^{x} g(s) \dif s - c                       \\
	\therefore F(x) & = \frac{f(x)}{2} - \frac{1}{2 a} \int\limits_{0}^{x} g(s) \dif s - \frac{c}{2} \\
\end{align*}
Therefore,
\begin{align*}
	u(x,t) & = F(x - a t) + G(x + a t)                                                                        \\
               & = \quad \frac{f(x - a t)}{2} - \frac{1}{2 a} \int\limits_{0}^{x - a t} g(s) \dif s - \frac{c}{2} \\
               & \quad + \frac{f(x + a t)}{2} + \frac{1}{2 a} \int\limits_{0}^{x + a t} g(s) \dif s + \frac{x}{2} \\
               & = \frac{f(x - a t) + f(x + a t)}{2} + \frac{1}{2 a} \int\limits_{x - a t}^{x + a t} g(s) \dif s
\end{align*}
\end{proof}

\subsection{Half-infinite Strings}

\begin{theorem}[Solution to initial boundary value problem for half-infinite string with fixed boundary]
	The solution to the initial boundary value problem
	\begin{align*}
		u_{t t}(x,t) &= a^2 u_{x x}(x,t)\\
		u(x,0) &= f(x)\\
		u_t(x,0) &= g(x)\\
		u(0,t) &= 0
	\end{align*}
	where $0 \le x < \infty$, $t \ge 0$ is
	\begin{align*}
		\tilde{u}(x,t) &= \frac{\tilde{f}(x - a t) + \tilde{f}(x + a t)}{2} + \frac{1}{2 a} \int\limits_{x - a t}^{x + a t} \tilde{g}(s) \dif s
	\end{align*}
	where
	\begin{align*}
		\tilde{f} &=
			\begin{cases}
				f(x) &;\quad x \ge 0\\
				-f(-x) &;\quad x < 0
			\end{cases}\\
		\tilde{g} &=
			\begin{cases}
				g(x) &;\quad x \ge 0\\
				-g(-x) &;\quad x < 0
			\end{cases}
	\end{align*}
	where $f$ is twice differentiable, $f(0) = 0$, $g$ is differentiable, and $g(0) = 0$.
	\label{thm:Solution_to_initial_boundary_value_problem_for_half-infinite_string_with_fixed_boundary}
\end{theorem}

\begin{proof}
	By the initial and boundary conditions,
	\begin{align*}
		u(x,0) &= f(x)\\
		\therefore u(0,0) &= f(0)\\
		u(0,t) &= 0\\
		\therefore u(0,0) &= 0\\
	\end{align*}
	Therefore,
	\begin{align*}
		\therefore f(0) &= 0
	\end{align*}
	Similarly,
	\begin{align*}
		u_t(x,0) &= g(x)\\
		\therefore u_t(0,0) &= g(0)\\
		u(0,t) &= 0\\
		\therefore u_t(0,t) &= 0\\
		\therefore u_t(0,0) &= 0
	\end{align*}
	Therefore,
	\begin{align*}
		g(0) &= 0
	\end{align*}
	These conditions are called compatibility conditions.\\
	Let
	\begin{align*}
		\tilde{f} &=
			\begin{cases}
				f(x) &;\quad x \ge 0\\
				-f(-x) &;\quad x < 0
			\end{cases}\\
		\tilde{g} &=
			\begin{cases}
				g(x) &;\quad x \ge 0\\
				-g(-x) &;\quad x < 0
			\end{cases}
	\end{align*}
	Therefore, due to the compatibility conditions, the odd extensions are continuous.\\
	Therefore, by the \nameref{thm:Solution_to_Cauchy_Problem},
	\begin{align*}
		\tilde{u}(x,t) &= \frac{\tilde{f}(x - a t) + \tilde{f}(x + a t)}{2} + \frac{1}{2 a} \int\limits_{x - a t}^{x + a t} \tilde{g}(s) \dif s
	\end{align*}
\end{proof}

\begin{theorem}[Solution to initial boundary value problem for half-infinite string with free boundary]
	The solution to the initial boundary value problem
	\begin{align*}
		u_{t t}(x,t) &= a^2 u_{x x}(x,t)\\
		u(x,0) &= f(x)\\
		u_t(x,0) &= g(x)\\
		u_x(0,t) &= 0
	\end{align*}
	where $0 \le x < \infty$, $t \ge 0$ is
	\begin{align*}
		\tilde{u}(x,t) &= \frac{\tilde{f}(x - a t) + \tilde{f}(x + a t)}{2} + \frac{1}{2 a} \int\limits_{x - a t}^{x + a t} \tilde{g}(s) \dif s
	\end{align*}
	where
	\begin{align*}
		\tilde{f} &=
			\begin{cases}
				f(x) &;\quad x \ge 0\\
				-f(-x) &;\quad x < 0
			\end{cases}\\
		\tilde{g} &=
			\begin{cases}
				g(x) &;\quad x \ge 0\\
				-g(-x) &;\quad x < 0
			\end{cases}
	\end{align*}
	where $f$ is twice differentiable, $f'(0) = 0$, $g$ is differentiable, and $g'(0) = 0$.
	\label{thm:Solution_to_initial_boundary_value_problem_for_half-infinite_string_with_free_boundary}
\end{theorem}

\begin{proof}
	By the initial and boundary conditions,
	\begin{align*}
		u(x,0) &= f(x)\\
		\therefore u_x(x,0) &= f'(x)\\
		\therefore u_x(0,0) &= f'(0)\\
		u_x(0,t) &= 0\\
		\therefore u_x(0,0) &= 0
	\end{align*}
	Therefore,
	\begin{align*}
		\therefore f'(0) &= 0
	\end{align*}
	Similarly,
	\begin{align*}
		u_t(x,0) &= g(x)\\
		\therefore u_{t x}(x,0) &= g'(x)\\
		\therefore u_{t x}(0,0) &= g'(0)\\
		u_x(0,t) &= 0\\
		\therefore u_{x t}(0,t) &= 0\\
		\therefore u_{x t}(0,0) &= 0
	\end{align*}
	Therefore,
	\begin{align*}
		g'(0) &= 0
	\end{align*}
	These conditions are called compatibility conditions.\\
	Let
	\begin{align*}
		\tilde{f} &=
			\begin{cases}
				f(x) &;\quad x \ge 0\\
				f(-x) &;\quad x < 0
			\end{cases}\\
		\tilde{g} &=
			\begin{cases}
				g(x) &;\quad x \ge 0\\
				g(-x) &;\quad x < 0
			\end{cases}
	\end{align*}
	Therefore, by the \nameref{thm:Solution_to_Cauchy_Problem},
	\begin{align*}
		\tilde{u}(x,t) &= \frac{\tilde{f}(x - a t) + \tilde{f}(x + a t)}{2} + \frac{1}{2 a} \int\limits_{x - a t}^{x + a t} \tilde{g}(s) \dif s
	\end{align*}
\end{proof}

\subsection{Finite Strings}

\begin{theorem}[Solution to boundary value problem for finite string with fixed boundary]
	The solution to the initial boundary value problem
	\begin{align*}
		u_{t t}(x,t) &= a^2 u_{x x}(x,t)\\
		u(x,0) &= f(x)\\
		u_t(x,0) &= g(x)\\
		u(0,t) &= 0\\
		u(l,t) &= 0
	\end{align*}
	where $0 \le x \le l$, $t \ge 0$ is
	\begin{align*}
		\tilde{u}(x,t) &= \frac{\tilde{f}(x - a t) + \tilde{f}(x + a t)}{2} + \frac{1}{2 a} \int\limits_{x - a t}^{x + a t} \tilde{g}(s) \dif s
	\end{align*}
	where $\tilde{f}$ and $\tilde{g}$ are the $2 l$ periodic extensions of the odd extensions of $f$ and $g$ respectively, where $f$ is twice differentiable, $f'(0) = 0$, $g$ is differentiable, and $g'(0) = 0$.
	\label{thm:Solution_to_initial_boundary_value_problem_for_finite_string_with_free_boundary}
\end{theorem}

\section{A Particular Case of Sturm-Liouville Problem}

Consider the problem
\begin{align*}
	X''(x) + \lambda X(x) &= 0\\
	X(0) &= 0\\
	X(l) &= 0
\end{align*}
on $[0,l]$.\\
Let $\lambda > 0$.
Therefore, let
\begin{align*}
	\lambda &= \omega^2
\end{align*}
where $\omega > 0$.\\
Therefore, the characteristic equation is
\begin{align*}
	r^2 + \omega^2 &= 0
\end{align*}
Therefore, solving,
\begin{align*}
	r &= \pm i \omega
\end{align*}
Therefore the solution of the ODE is
\begin{align*}
	X(s) &= A \cos(\omega x) + B \sin(\omega x)
\end{align*}
Therefore, substituting the given boundary conditions,
\begin{align*}
	X(0) &= 0\\
	\therefore A &= 0\\
	X(l) &= 0\\
	\therefore B \sin(\omega l) &= 0\\
	\therefore \sin(\omega l) &= 0\\
	\therefore \omega l &= n \pi
\end{align*}
where $n \in \mathbb{N}$.
\marginnote
{
	If $n \in \mathbb{Z}$, then $\omega \le 0$.
	This contradicts the assumption $\omega > 0$.
}
\begin{align*}
	\lambda_n &= {\omega_n}^2\\
	&= \left( \frac{n \pi}{l} \right)^2
\end{align*}
where $n \in \mathbb{N}$, is called an eigenvalue of the problem.\\
The corresponding solution to the problem is
\begin{align*}
	X_n &= B_n \sin\left( \frac{n \pi}{l} x \right)
\end{align*}
The function
\begin{align*}
	X_n &= \sin\left( \frac{n \pi}{l} x \right)
\end{align*}
is called an eigenfunction of the problem, corresponding to the eigenvalue $\lambda_n$.

\begin{question}
	Solve
	\begin{align*}
		u_{t t} &= u_{x x}\\
		u(x,0) &= x (2 - x)\\
		u_t(x,0) &= 0\\
		u(0,t) &= 0
	\end{align*}
	where $x > 0$, $t > 0$.
\end{question}

\begin{solution}
	Comparing to the standard form,
	\begin{align*}
		a &= 1\\
		f(x) &= x (2 - x)\\
		g(x) &= 0
	\end{align*}
	Therefore, as the boundary is fixed, let
	\begin{align*}
		\tilde{f} &=
			\begin{cases}
				f(x) &;\quad x \ge 0\\
				-f(-x) &;\quad x < 0
			\end{cases}\\
		&=
			\begin{cases}
				x (2 - x) &;\quad x \ge 0\\
				- \left( -x (2 + x) \right) &;\quad x < 0\\
			\end{cases}\\
		\tilde{g} &=
			\begin{cases}
				g(x) &;\quad x \ge 0\\
				-g(-x) &;\quad x < 0
			\end{cases}\\
		&= 0
	\end{align*}
	Therefore,
	\begin{align*}
		\tilde{u}(x,t) &= \frac{\tilde{f}(x - a t) + \tilde{f}(x + a t)}{2} + \frac{1}{2 a} \int\limits_{x - a t}^{x + a t} \tilde{g}(s) \dif s\\
		&= \frac{\tilde{f}(x - t) + \tilde{f}(x + t)}{2}\\
		&= \quad
			\begin{cases}
				\frac{1}{2} (x - t) \left( 2 - (x - t) \right) &;\quad x - t \ge 0\\
				\frac{1}{2} (x - t) \left( 2 + (x - t) \right) &;\quad x - t < 0\\
			\end{cases}\\
		&\quad +
			\begin{cases}
				\frac{1}{2} (x + t) \left( 2 - (x + t) \right) &;\quad x - t \ge 0\\
				\frac{1}{2} (x + t) \left( 2 + (x + t) \right) &;\quad x - t < 0\\
			\end{cases}\\
		&= \quad
			\begin{cases}
				\frac{1}{2} (x - t) \left( 2 - (x - t) \right) &;\quad x \ge t\\
				\frac{1}{2} (x - t) \left( 2 + (x - t) \right) &;\quad x < t\\
			\end{cases}\\
		&\quad +
			\begin{cases}
				\frac{1}{2} (x + t) \left( 2 - (x + t) \right) &;\quad x \ge -t\\
				\frac{1}{2} (x + t) \left( 2 + (x + t) \right) &;\quad x < -t\\
			\end{cases}
	\end{align*}
	Therefore, the restricted solution, i.e. the solution on the given domain $x > 0$, $t > 0$ is
	\begin{align*}
		u(x,t) &=
			\begin{cases}
				\frac{1}{2} \left( (x + t) (2 - x - t) + (x - t) (2 + x - t) \right) &;\quad 0 < x < t\\
				\frac{1}{2} \left( (x + t) (2 - x - t) + (x - t) (2 - x + t) \right) &;\quad t \le x\\
			\end{cases}
	\end{align*}
\end{solution}

\begin{question}
	Solve
	\begin{align*}
		u_{t t} &= 2 u_{x x}\\
		u(x,0) &= x^2\\
		u_t(x,0) &= \sin x\\
		u_x(0,t) &= 0
	\end{align*}
	where $x > 0$, $t > 0$.
\end{question}

\begin{solution}
	Comparing to the standard form,
	\begin{align*}
		a &= 2\\
		f(x) &= x^2\\
		g(x) &= \sin x
	\end{align*}
	Therefore, as the boundary is free, let
	\begin{align*}
		\tilde{f} &=
			\begin{cases}
				f(x) &;\quad x \ge 0\\
				f(-x) &;\quad x < 0
			\end{cases}\\
		&=
			\begin{cases}
				x^2 &;\quad x \ge 0\\
				(-x)^2 &;\quad x < 0\\
			\end{cases}\\
		&= x^2\\
		\tilde{g} &=
			\begin{cases}
				g(x) &;\quad x \ge 0\\
				g(-x) &;\quad x < 0
			\end{cases}\\
		&=
			\begin{cases}
				\sin x &;\quad x \ge 0\\
				\sin(-x) &;\quad x < 0\\
			\end{cases}\\
		&= \sin|x|
	\end{align*}
	Therefore,
	\begin{align*}
		\tilde{u}(x,t) &= \frac{\tilde{f}(x - a t) + \tilde{f}(x + a t)}{2} + \frac{1}{2 a} \int\limits_{x - a t}^{x + a t} \tilde{g}(s) \dif s\\
		&= \frac{(x - 2 t)^2 + (x + 2 t)^2}{2} + \frac{1}{4} \int\limits_{x - 2 t}^{x + 2 t} \sin|s| \dif s\\
		&= \frac{2 x^2 + 8 t^2}{2} +  \frac{1}{4} \int\limits_{x - 2 t}^{x + 2 t} \sin|s| \dif s\\
		&= x^2 + 4 t^2 + \frac{1}{4} \int\limits_{x - 2 t}^{x + 2 t} \sin|s| \dif s
	\end{align*}
	Therefore, the restricted solution, i.e. the solution on the given domain $x > 0$, $t > 0$ is
	\begin{align*}
		u(x,t) &= x^2 + 4 t^2 + 
			\begin{cases}
				\frac{1}{4} \int\limits_{x - 2 t}^{x + 2 t} \sin s \dif s &;\quad x - 2 t \ge 0\\
				\frac{1}{4} \int\limits_{x - 2 t}^{0} \sin(-s) \dif s + \frac{1}{4} \int\limits_{0}^{x + 2 t} \sin s \dif s &;\quad x - 2 t < 0\\
			\end{cases}\\
		&= x^2 + 4 t^2 +
			\begin{cases}
				\frac{1}{4} \left( \cos(x - 2 t) - \cos(x + 2 t) \right) &;\quad x \ge 2 t\\
				\frac{1}{4} \left( 2 \cos(0) - \cos(x - 2 t) - \cos(x + 2 t) \right) &;\quad 0 < x < 2 t\\
			\end{cases}\\
		&= x^2 + 4 t^2 +
			\begin{cases}
				\frac{1}{4} \left( 2 \sin x \sin(2 t) \right) &;\quad x \ge 2 t\\
				\frac{1}{4} \left( 2 - 2 \cos x \cos(2 t) \right) &;\quad 0 < x < 2 t\\
			\end{cases}\\
		&=
			\begin{cases}
				x^2 + 4 t^2 + \frac{1}{2} \sin x \sin(2 t) &;\quad x \ge 2 t\\
				x^2 + 4 t^2 + \frac{1}{2} \left( 1 - \cos x \cos(2 t) \right) &;\quad 0 < x < 2 t\\
			\end{cases}
	\end{align*}
	In this case, even though $g'(0) \neq 0$, the calculated solution is a valid solution for the problem.
\end{solution}

\end{document}
