\documentclass[titlepage, fleqn, a4paper, 12pt, twoside]{article}
\usepackage{geometry}
\usepackage{exsheets} %question and solution environments
\usepackage{amsmath, amssymb, amsthm} %standard AMS packages
\usepackage{esint} %integral signs
\usepackage{marginnote} %marginnotes
\usepackage{gensymb} %miscellaneous symbols
\usepackage{commath} %differential symbols
\usepackage{xcolor} %colours
\usepackage{cancel} %cancelling terms
\usepackage[free-standing-units]{siunitx} %formatting units
\usepackage{tikz, pgfplots} %diagrams
	\usetikzlibrary{calc, hobby, patterns, intersections, angles, quotes, spy}
\usepackage{graphicx} %inserting graphics
\usepackage{epstopdf} %converting and inserting eps graphics
\usepackage{hyperref} %hyperlinks
\usepackage{datetime} %date and time
\usepackage{enumerate, enumitem} %numbered lists
\usepackage{float} %inserting floats
\usepackage{microtype} %micro-typography
\usepackage{todonotes}
\usepackage{booktabs}

\newcommand\numberthis{\addtocounter{equation}{1}\tag{\theequation}} %adds numbers to specific equations in non-numbered list of equations

\theoremstyle{definition}
\newtheorem{example}{Example}
\newtheorem{definition}{Definition}

\theoremstyle{theorem}
\newtheorem{theorem}{Theorem}
\newtheorem{law}{Law}

\makeatletter
\@addtoreset{section}{part} %resets section numbers in new part
\makeatother

\newcommand\blfootnote[1]{%
	\begingroup
	\renewcommand\thefootnote{}\footnote{#1}%
	\addtocounter{footnote}{-1}%
	\endgroup
}

\renewcommand{\marginfont}{\scriptsize \color{blue}}

\renewcommand{\tilde}{\widetilde}

\SetupExSheets{solution/print = true} %prints all solutions by default

%opening
\title{Partial Differential Equations}
\author{Aakash Jog}
\date{2015-16}

\begin{document}

\pagenumbering{roman}
\begin{titlepage}
\newgeometry{margin=0cm}
\maketitle
\end{titlepage}
\restoregeometry
%\setlength{\mathindent}{0pt}

\blfootnote
{	
	\begin{figure}[H]
		\includegraphics[height = 12pt]{cc.pdf}
		\includegraphics[height = 12pt]{by.pdf}
		\includegraphics[height = 12pt]{nc.pdf}
		\includegraphics[height = 12pt]{sa.pdf}
	\end{figure}
	This work is licensed under the Creative Commons Attribution-NonCommercial-ShareAlike 4.0 International License. To view a copy of this license, visit \url{http://creativecommons.org/licenses/by-nc-sa/4.0/}.
} %CC-BY-NC-SA license

\tableofcontents

\clearpage
\section{Lecturer Information}

\textbf{Prof. Yakov Yakubov}\\
~\\
Office: Schreiber 233\\
E-mail: \href{mailto:yakubov@post.tau.ac.il}{yakubov@post.tau.ac.il}\\

\section{Recommended Reading}

\begin{enumerate}
	\item Tikhonov, A.N. and Samarskii, N.A: Equations of Mathematical Physics, Pergamon Press, Oxfort, 1963.
	\item Weinberger, H.F, A first Course in Partial Differential Equations, Dover, NY, 1995.
\end{enumerate}

\clearpage
\pagenumbering{arabic}

\part{Introduction}

\section{String Equations}

\begin{definition}[Partial differential equation]
	An equation
	\begin{align*}
		F(x_1,x_2,\dots,u,u_{x_1},u_{x_2},\dots,u_{x_1 x_2},\dots) & = 0
	\end{align*}
	where all $x_i$ are independent variables, and $u(x_1,..,x_n)$ is an unknown function, is called a partial differential equation.\\
	A partial differential equation describes a connection between an unknown function of several variables and its partial derivatives.
\end{definition}

\begin{definition}[Order of a PDE]
	The order of a PDE is defined to be the highest order of partial derivatives in the equation.
\end{definition}

\begin{definition}[Linear PDE]
	A PDE is said o be linear if and only if it is a linear function of $u$ and its partial derivatives.
\end{definition}

\begin{definition}[String equation/1D Wave Equation]
	Consider an ideal string on the $x$-axis.
	Let the string oscillate in the direction normal to the $x$-axis.
	Let $u$ be the position function of a point on the string.
	Therefore, $u$ depends on the position of the point on the string and on the time, i.e. it is a function of $x$ and $t$.\\
	Therefore, solving using Newton's Laws,
	\begin{align*}
		\rho(x) u_{t t}(x,t) & = T u_{x x}(x,t)
	\end{align*}
	where $\rho$ is the mass density of the string, and $T$ is the tension in the string.\\
	If
	\begin{align*}
		\rho(x_0) & = \rho_0
	\end{align*}
	then,
	\begin{align*}
		u_{t t}(x,t) & = a^2 u_{x x}(x,t)
	\end{align*}
	where
	\begin{align*}
		a & = \sqrt{\frac{T}{\rho_0}}
	\end{align*}
	If there is an external force applied to the string,
	\begin{align*}
		\rho(x) u_{t t}(x,t) & = a^2 u_{x x}(x,t) + F(x,t)
	\end{align*}
\end{definition}

\begin{definition}[Cauchy problem]
	Consider an infinite string, i.e. $x \in (-\infty,\infty)$.
	If the initial position and the initial velocity of the string are given to be $f(x)$ and $g(x)$ respectively, then,
	\begin{align*}
		u(x,0)   & = f(x) \\
		u_x(x,0) & = g(x)
	\end{align*}
	The problem
	\begin{align*}
		u_{t t}(x,t) & = a^2 u_{x x}(x,t) \\
		u(x,0)       & = f(x)             \\
		u_x(x,0)     & = g(x)
	\end{align*}
	is called the Cauchy problem.
\end{definition}

\begin{definition}[Dirichlet's boundary conditions]
	Consider a finite string, such that $x \in [0,l]$.
	If the ends of the string are fixed, the boundary conditions
	\begin{align*}
		u(0,t) & = 0 \\
		u(l,t) & = 0
	\end{align*}
	are called Dirichlet's boundary conditions.
\end{definition}

\begin{definition}[General string equation]
	Consider a PDE
	\begin{align*}
		u_{t t}(x,t) & = a^2 u_{x x}(x,t)
	\end{align*}
	Let
	\begin{align*}
		\zeta & = x - a t \\
		\eta  & = x + a t
	\end{align*}
	Therefore,
	\begin{align*}
		u(x,t) & = F(\zeta) + G(\eta) \\
                       & = F(x - a t) + G(x + a t)
	\end{align*}
	where $F$ and $G$ are functions of a single variable, and are differentiable twice.
\end{definition}

\subsection{Solution to Cauchy Problem}

\begin{theorem}
	The solution to the Cauchy problem
	\begin{align*}
		u_{t t}(x,t) & = a^2 u_{x x}(x,t) \\
		u(x,0)       & = f(x)             \\
		u_x(x,0)     & = g(x)
	\end{align*}
	is given by the d'Alembert formula, i.e.
	\begin{align*}
		u(x,t) & = \frac{f(x - a t) + f(x + a t)}{2} + \frac{1}{2 a} \int\limits_{x - a t}^{x + a t} g(s) \dif s
	\end{align*}
	where $f$ is twice differentiable and $g$ is differentiable.
\end{theorem}

\begin{align*}
	u_{t t}(x,t) & = a^2 u_{x x}(x,t) \\
	u(x,0)       & = f(x)             \\
	u_x(x,0)     & = g(x)
\end{align*}
Let the solution be
\begin{align*}
	u(x,t) & = F(x - a t) + G(x + a t)
\end{align*}
Therefore,
\begin{align*}
	u_t(x,t) & = \dod{u(x,t)}{(x - a t)} \dod{(x - a t)}{t} \\
                 & = F'(x - a t) (-a) + G'(x + a t) (a)         \\
                 & = -a F'(x - a t) + a G'(x + a t)
\end{align*}
Substituting the initial conditions,
\begin{align*}
	u(x,0)   & = f(x)        \\
                 & = F(x) + G(x) \\
	u_t(x,0) & = g(x)        \\
                 & = -a F'(x) + a G'(x)
\end{align*}
Therefore,
\begin{align*}
	a \int\limits_{0}^{x} \left( -F'(s) + G'(s) \right) \dif s & = \int\limits_{0}^{x} g(s) \dif s \\
	\therefore -F(x) + G(x)                                    & = \frac{1}{a} \int\limits_{0}^{x} g(s) \dif s + c
\end{align*}
Therefore, solving with the initial conditions corresponding to $u(x,0)$,
\begin{align*}
	2 G(x)          & = f(x) + \frac{1}{a} \int\limits_{0}^{x} g(s) \dif s + c                       \\
	\therefore G(x) & = \frac{f(x)}{2} + \frac{1}{2 a} \int\limits_{0}^{x} g(s) \dif s + \frac{c}{2} \\
	2 F(x)          & = f(x) - \frac{1}{a} \int\limits_{0}^{x} g(s) \dif s - c                       \\
	\therefore F(x) & = \frac{f(x)}{2} - \frac{1}{2 a} \int\limits_{0}^{x} g(s) \dif s - \frac{c}{2} \\
\end{align*}
Therefore,
\begin{align*}
	u(x,t) & = F(x - a t) + G(x + a t)                                                                        \\
               & = \quad \frac{f(x - a t)}{2} - \frac{1}{2 a} \int\limits_{0}^{x - a t} g(s) \dif s - \frac{c}{2} \\
               & \quad + \frac{f(x + a t)}{2} + \frac{1}{2 a} \int\limits_{0}^{x + a t} g(s) \dif s + \frac{x}{2} \\
               & = \frac{f(x - a t) + f(x + a t)}{2} + \frac{1}{2 a} \int\limits_{x - a t}^{x + a t} g(s) \dif s
\end{align*}

\end{document}
