\documentclass[fleqn, a4paper, 11pt, oneside]{amsart}
\usepackage{exsheets}
\usepackage{amsmath, amssymb, amsthm} %standard AMS packages
\usepackage{marginnote} %marginnotes
\usepackage{gensymb} %miscellaneous symbols
\usepackage{commath} %differential symbols
\usepackage{xcolor} %colours
\usepackage{cancel} %cancelling terms
\usepackage[free-standing-units, space-before-unit]{siunitx} %formatting units
\usepackage{tikz, pgfplots} %diagrams
	\usetikzlibrary{calc, hobby, patterns, intersections, decorations.markings}
\usepackage{graphicx} %inserting graphics
\usepackage{hyperref} %hyperlinks
\usepackage{datetime} %date and time
\usepackage{enumerate,enumitem} %numbered lists
\usepackage{float} %inserting floats
\usepackage{circuitikz}[american voltages, american currents] %circuit diagrams

\newcommand\numberthis{\addtocounter{equation}{1}\tag{\theequation}} %adds numbers to specific equations in non-numbered list of equations

\theoremstyle{definition}
\newtheorem{example}{Example}
\newtheorem{definition}{Definition}

\theoremstyle{theorem}
\newtheorem{theorem}{Theorem}

\makeatletter
\@addtoreset{section}{part} %resets section numbers in new part
\makeatother

\renewcommand{\tilde}{\widetilde}

\SetupExSheets{solution/print = true}

%opening
\title
[
	PDE : Assignment 4
]
{
	Partial Differential Equations\\
	Assignment 4
}
\author
{
	Aakash Jog\\
	ID : 989323563
}
\date{\formatdate{31}{3}{2016}}

\begin{document}

\maketitle
%\setlength{\mathindent}{0pt}

\begin{question}
	Using the energy method, prove uniqueness of the solution to the Dirichlet problem of the heat equation
	\begin{align*}
		u_t - k u_{x x} &= F(x,t)\\
		u(0,t) &= a(t)\\
		u(L,t) &= b(t)\\
		u(x,0) &= f(x)
	\end{align*}
	where $0 \le x \le L$, $t \ge 0$, and $k$ is a positive constant.\\
	Hint: Show that the energy
	\begin{align*}
		E(t) &= \frac{1}{2} \int\limits_{0}^{L} w^2 \dif x
	\end{align*}
	satisfies
	\begin{align*}
		E'(t) &\le 0
	\end{align*}
	where $w$ is a solution of the homogeneous problem.
\end{question}

\begin{solution}
	If possible, let $u_1$ and $u_2$ be two distinct solutions of the problem.\\
	Let
	\begin{align*}
		v(x,t) &= u_1(x,t) - u_2(x,t)
	\end{align*}
	Therefore,
	\begin{align*}
		v_t - k v_{x x} &= 0\\
		v(0,t) &= 0\\
		v(L,t) &= 0\\
		v(x,0) &= 0
	\end{align*}
	Therefore,
	\begin{align*}
		E(t) &= \frac{1}{2} \int\limits_{0}^{L} v^2 \dif x\\
		\therefore E'(t) &= \frac{1}{2} \int\limits_{0}^{L} 2 v v_t \dif x\\
		&= \int\limits_{0}^{L} k v v_{x x} \dif x\\
		&= k \int\limits_{0}^{L} v v_{x x} \dif x\\
		&= \left. k v \int v_{x x} \dif x - k \int v_x \int v_{x x} \dif x \right|_{0}^{L}\\
		&= \left. k v v_x \right|_{0}^{L} - \int\limits_{0}^{L} k {v_x}^2 \dif x\\
		&= -\int\limits_{0}^{L} {v_x}^2 \dif x\\
		&\le 0
	\end{align*}
	Also,
	\begin{align*}
		E(t) &= \frac{1}{2} \int\limits_{0}^{L} v^2 \dif x\\
		&\ge 0
	\end{align*}
	Therefore,
	\begin{align*}
		E(t) &\equiv 0
	\end{align*}
	Therefore,
	\begin{align*}
		\frac{1}{2} \int\limits_{0}^{L} v^2 \dif x &\equiv 0\\
		\therefore v &\equiv 0
	\end{align*}
	Therefore,
	\begin{align*}
		u_1 &\equiv u_2
	\end{align*}
	This contradicts the assumption that $u_1$ and $u_2$ are distinct.
	Therefore, the problem has a unique solution.
\end{solution}

\begin{question}
	Using the energy method, prove uniqueness of the solution to the following problem of the string equation
	\begin{align*}
		u_{t t} - u_{x x} &= x t\\
		u_x(0,t) &= g(t)\\
		u(1,t) &= h(t)\\
		u(x,0) &= x^2 - 1\\
		u_t(x,0) &= x^{2016} - 1
	\end{align*}
	where $0 \le x \le 1$, $t \ge 0$.
\end{question}

\begin{solution}
	If possible, let $u_1$ and $u_2$ be two distinct solutions of the problem.\\
	Let
	\begin{align*}
		v(x,t) &= u_1(x,t) - u_2(x,t)
	\end{align*}
	Therefore,
	\begin{align*}
		v_{t t} - v_{x x} &= 0\\
		v_x(0,t) &= 0\\
		v(1,t) &= 0\\
		v(x,0) &= 0\\
		v_t(x,0) &= 0
	\end{align*}
	Therefore, comparing to the standard form,
	\begin{align*}
		\rho(t) &= 1\\
		k(x) &= 1
	\end{align*}
	Therefore,
	\begin{align*}
		E(t) &= \frac{1}{2} \int\limits_{0}^{l} \left( k {v_x}^2 + \rho {v_t}^2 \right) \dif x\\
		&= \frac{1}{2} \int\limits_{0}^{1} {v_x}^2 + {v_t}^2 \dif x\\
		\therefore E'(t) &= \frac{1}{2} \int\limits_{0}^{1} 2 v_x v_{x t} + 2 v_t v_{t t} \dif x\\
		&= \int\limits_{0}^{1} v_x v_{x t} \dif t + \int\limits_{0}^{1} v_t v_{t t} \dif x
	\end{align*}
	Assuming the mixed derivatives exist and are continuous,
	\begin{align*}
		v_{x t} &= v_{t x}
	\end{align*}
	Therefore,
	\begin{align*}
		E'(t) &= \int\limits_{0}^{1} v_x v_{t x} \dif x + \int\limits_{0}^{1} v_t v_{t t} \dif x\\
		&= \int\limits_{0}^{1} v_x v_{t t} \dif x + \left. v_x v_t \right|_{0}^{1} - \int\limits_{0}^{1} v_t v_{x x} \dif x\\
		&= \left. v_x v_t \right|_{0}^{1}\\
		&\equiv 0
	\end{align*}
	Therefore,
	\begin{align*}
		E(t) &\equiv c
	\end{align*}
	Therefore, as $v(x,0) = 0$,
	\begin{align*}
		v_x(x,0) &= 0
	\end{align*}
	Also,
	\begin{align*}
		v_t(x,0) &= 0
	\end{align*}
	Therefore,
	\begin{align*}
		E(t) &\equiv 0
	\end{align*}
	Therefore,
	\begin{align*}
		v_x &\equiv v_t\\
		&\equiv 0
	\end{align*}
	Therefore,
	\begin{align*}
		u_1 &\equiv u_2
	\end{align*}
	This contradicts the assumption that $u_1$ and $u_2$ are distinct.
	Therefore, the problem has a unique solution.
\end{solution}

\begin{question}
	Using the energy method, prove uniqueness of the solution to the Neumann problem of the string equation
	\begin{align*}
		u_{t t} - 4 u_{x x} &= x t\\
		u(x,0) &= \cos^2(\pi x)\\
		u_t(x,0) &= \sin^2(\pi x) \cos(\pi x)\\
		u_x(0,t) &= 0\\
		u_x(1,t) &= 0
	\end{align*}
	where $0 \le x \le 1$, $t \ge 0$.
\end{question}

\begin{solution}
	If possible, let $u_1$ and $u_2$ be two distinct solutions of the problem.\\
	Let
	\begin{align*}
		v(x,t) &= u_1(x,t) - u_2(x,t)
	\end{align*}
	Therefore,
	\begin{align*}
		v_{t t} - 4 v_{x x} &= 0\\
		v(x,0) &= 0\\
		v_t(x,0) &= 0\\
		v_x(0,t) &= 0\\
		v_x(1,t) &= 0
	\end{align*}
	Therefore, comparing to the standard form,
	\begin{align*}
		\rho(t) &= 1\\
		k(t) &= 4
	\end{align*}
	Therefore,
	\begin{align*}
		E(t) &= \frac{1}{2} \int\limits_{0}^{l} \left( k {v_x}^2 + \rho {v_t}^2 \right) \dif x\\
		&= \frac{1}{2} \int\limits_{0}^{1} 4 {v_x}^2 + {v_t}^2 \dif x\\
		\therefore E'(t) &= \frac{1}{2} \int\limits_{0}^{1} 8 v_x v_{x t} + 2 v_t v_{t t} \dif x\\
		&= \int\limits_{0}^{1} 4 v_x v_{x t} \dif x + \int\limits_{0}^{1} v_t v_{t t} \dif x
	\end{align*}
	Assuming the mixed derivatives exist and are continuous,
	\begin{align*}
		v_{x t} &= v_{t x}
	\end{align*}
	Therefore,
	\begin{align*}
		E'(t) &= \int\limits_{0}^{1} 4 v_x v_{t x} \dif x + \int\limits_{0}^{1} v_t v_{t t} \dif x\\
		&= \int\limits_{0}^{1} v_t v_{t t} \dif x + \left. 4 v_x v_t \right|_{0}^{1} - \int\limits_{0}^{1} 4 v_t v_{x x} \dif x\\
		&= \left. 4 v_x v_t \right|_{0}^{1}\\
		&\equiv 0
	\end{align*}
	Therefore,
	\begin{align*}
		E(t) &\equiv c
	\end{align*}
	Therefore, as $v(x,0) = 0$,
	\begin{align*}
		v_x(x,0) &= 0
	\end{align*}
	Also,
	\begin{align*}
		v_t(x,0) &= 0
	\end{align*}
	Therefore,
	\begin{align*}
		E(t) &\equiv 0
	\end{align*}
	Therefore,
	\begin{align*}
		v_x &\equiv v_t\\
		&\equiv 0
	\end{align*}
	Therefore,
	\begin{align*}
		u_1 &\equiv u_2
	\end{align*}
	This contradicts the assumption that $u_1$ and $u_2$ are distinct.
	Therefore, the problem has a unique solution.
\end{solution}

\end{document}
